\documentclass[a4paper]{scrartcl}
% this is a comment

%opening
\title{A great test document}
\subtitle{For testing proposes}
\author{esoares}

% importing packages
\usepackage{acronym}
\usepackage{url}

% start of text that will be actually displayed
\begin{document}

\maketitle

\begin{abstract}
% phrases could be multiline parted, new line is marked with \newline, \\ or double line change
Testing the
test\ldots

An actual new paragraph.
\end{abstract}

\section{Introduction}

% notice the "ac" command, this a modifier of the inside text created by the imported package "acronym"
% packages can add new commands
% this particular adds acronyms, which means that the first occurrence of the acronym will be expanded
% acronyms need to be defined somewhere, with another command (\acrodef), but I will keep it out of the scope of this example
\section{Well, this is \ac{ok}}

Let's add a figure! Check out figure~\ref{fig:figure1}.

% begining of a special environment to display a image
\begin{figure}
	\includegraphics[width=0.9\linewidth, page=5]{yoda.pdf}
	\caption{good caption it is}
	\label{fig:figure1}
\end{figure}

% special case, where some modifiers to the text can be performed between {\modifier }
Writing some really tiny text with {\tiny this is a tiny text}.

\section{Conclusion}

Done?
Wait, lets cite some work~\cite{bibliography-mark}.

% where can be put some extra commands to generate a bibliography for instance

\end{document}
